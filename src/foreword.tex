\chapter*{前言和致谢}

这是一篇用于操作系统课程的文章草案。它通过学习一个名为xv6的示例内核来解释操作系统的核心概念。xv6是Dennis Ritchie和Ken Thompson的Unix Version 6(v6)的复现。xv6基本上遵循了v6的架构和风格,不过是基于x86架构的多核处理器的ANSI C的实现。

这篇文章应该和xv6的源码一起阅读。这种方式由John Lions的《Commentary on UNIX》第6版(Peer to Peer Communications; ISBN: 1-57398-013-7; 1st edition (June 14, 2000))最先提出。见\url{https://pdos.csail.mit.edu/6.828}获取v6和xv6的在线资源,包括一些使用xv6的实操作业。

我们在6.828中使用了这篇文章,它是MIT的操作系统课程。我们感谢6.828的教职员工、助教和学生,他们都直接或间接地为xv6做出过贡献。我们还特别想感谢Austin Clements和Nickolai Zeldovich。最后,我们想感谢所有通过邮件告诉我们书中的bug或者改进的人:Abutalib Aghayev, Sebastian Boehm, Anton Burtsev, Raphael Carvalho, Rasit Eskicioglu, Color Fuzzy, Giuseppe, Tao Guo, Robert Hilderman, Wolfgang Keller, Austin Liew, Pavan Maddamsetti, Jacek Masiulaniec, Michael McConville, miguelgvieira, Mark Morrissey, Harry Pan, Askar Safin, Salman Shah, Ruslan Savchenko, Pawel Szczurko, Warren Toomey, tyfkda, Zou Chang Wei.

如果你发现了错误或者有任何改进的建议,请给Frans Kaashoek和Robert Morris发送邮件(kaashoek, \url{rtm@csail.mit.edu})。
