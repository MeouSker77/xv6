\chapter{中断和设备驱动}

操作系统中的\emph{驱动(driver)}是指用来管理特定设备的代码:它需要配置设备硬件、通知硬件进行操作、处理结果中断、与那些等待这个设备I/O的进程交互。
驱动的代码可能很棘手,因为驱动要和它管理的设备并行执行。
另外,驱动必须理解设备的硬件接口,这些接口可能很复杂并且没有很好的文档。

需要操作系统关注的设备通常会被设置为可以产生中断,它是一种自陷。
当设备引发中断时,内核自陷的处理代码可以识别出来并调用驱动的中断处理程序,在xv6,这个分发操作在\texttt{devintr}\href{https://github.com/mit-pdos/xv6-riscv/blob/riscv//kernel/trap.c#L178}{(kernel/trap.c:178)}中。

很多设备驱动在两个上下文中执行代码:\emph{上半段(top half)}在进程的内核线程中执行,\emph{下半段(bottom half)}在中断时执行。
上半段由像\texttt{read}和\texttt{write}这样的希望设备进行I/O的系统调用进行调用。
这样的代码可能会要求硬件开始一个操作(例如,让磁盘读取一个块),然后代码会等待操作结束。
最后设备完成操作之后会产生一个中断。
驱动的中断处理程序,会进行下半段操作:查明什么操作结束了,如果需要的话唤醒一个正在等待的进程,并且如果有正在等待的下一个操作的话告诉硬件开始执行。


