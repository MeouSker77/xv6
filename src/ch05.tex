\chapter{中断和设备驱动}

操作系统中的\emph{驱动(driver)}是指用来管理特定设备的代码:它需要配置设备硬件、通知硬件进行操作、处理结果中断、与那些等待这个设备I/O的进程交互。
驱动的代码可能很棘手,因为驱动要和它管理的设备并行执行。
另外,驱动必须理解设备的硬件接口,这些接口可能很复杂并且没有很好的文档。

需要操作系统关注的设备通常会被设置为可以产生中断,它是一种自陷。
当设备引发中断时,内核自陷的处理代码可以识别出来并调用驱动的中断处理程序,在xv6,这个分发操作在\texttt{devintr}\href{https://github.com/mit-pdos/xv6-riscv/blob/risc/kernel/trap.c#L178}{(kernel/trap.c:178)}中。

很多设备驱动在两个上下文中执行代码:\emph{上半段(top half)}在进程的内核线程中执行,\emph{下半段(bottom half)}在中断时执行。
上半段由像\texttt{read}和\texttt{write}这样的希望设备进行I/O的系统调用进行调用。
这样的代码可能会要求硬件开始一个操作(例如,让磁盘读取一个块),然后代码会等待操作结束。
最后设备完成操作之后会产生一个中断。
驱动的中断处理程序,会进行下半段操作:查明什么操作结束了,如果需要的话唤醒一个正在等待的进程,并且如果有正在等待的下一个操作的话告诉硬件开始执行。

\section{代码:控制台输入}
控制台驱动\href{https://github.com/mit-pdos/xv6-riscv/blob/risc/kernel/console.c}{(kernel/console.c)}是一个驱动的简单示例。
控制台驱动通过连接到RISC-V\texttt{UART}串口硬件接受人类输入的字符。
控制台驱动一次累计一行输入,处理像退格和control-u这样的特殊输入字符。
用户进程,例如shell,使用\texttt{read}系统调用从控制台获取输入的行。
当你在QEMU中向xv6输入时,你的按键被QEMU模拟的UART硬件发送给xv6。

这个驱动交互的UART硬件是一个QEMU模拟出的16550芯片[13]。
在真实的计算机中,16550将管理连接到终端或其他计算机的RS232串行链路。
当运行QEMU时,它连接到你的键盘和显示器。

UART硬件以一组\emph{内存映射(memory-mapped)}控制寄存器的形式暴露给软件。
也就是说,有一些物理地址是连接到UART设备的,因此load和store操作会和设备进行交互,而不是和RAM交互。
UART的内存映射地址从0x10000000开始,或者叫\texttt{UART0}\href{https://github.com/mit-pdos/xv6-riscv/blob/risc/kernel/memlayout.h#L21}{(kernel/memlayout.h:21)}。
有几个UART的控制寄存器,每一个宽度都是一个字节。
它们相对于\texttt{UART0}的偏移量被定义在\href{https://github.com/mit-pdos/xv6-riscv/blob/risc/kernel/uart.c#L22}{(kernel/uart.c:22)}。
例如,\texttt{LSR}寄存器包含指示输入字符是否正在等待被软件读取的位。
这些字符(如果有的话)可以从\texttt{RHR}寄存器读取。
每被读取一个字符,UART硬件会把它从内部用于保存待读取字符的FIFO中删除,并且当FIFO为空时清除\texttt{LSR}中的“ready”位。
UART传输硬件在很大程度上独立于接收硬件,如果软件向\texttt{THR}中写入一个字节,UART就会传输这个字节。

xv6的\texttt{main}调用\texttt{consoleinit}\href{https://github.com/mit-pdos/xv6-riscv/blob/risc/kernel/console.c#L182}{(kernel/console.c:182)}来初始化UART硬件。
这段代码配置UART硬件在接收到每个输入字节时就产生一个接收中断,在每结束发送一个字节的输出时产生\emph{传输结束(transmit complete)}中断\href{https://github.com/mit-pdos/xv6-riscv/blob/risc/kernel/uart.c#L53}{(kernel/uart.c:53)}。

xv6的shell通过\texttt{init.c}\href{https://github.com/mit-pdos/xv6-riscv/blob/risc/user/init.c#L19}{(user/init.c:19)}打开的文件描述符从控制台读取输入。
\texttt{read}系统调用会为\texttt{consoleread}\href{https://github.com/mit-pdos/xv6-riscv/blob/risc/kernel/console.c#L80}{(kernel/console.c:80)}进行准备。
\texttt{consoleread}等待输入到达(通过中断)并被缓存到\texttt{cons.buf}中,然后把输入拷贝到用户空间,并(在接收到一整行后)返回到用户进程。
如果用户还没有输入完一行,任何读取操作都会在\texttt{sleep}调用处等待\href{https://github.com/mit-pdos/xv6-riscv/blob/risc/kernel/console.c#L96}{(kernel/console.c:96)}(\autoref{ch07}将解释\texttt{sleep}的细节)。

当用户输入一个字符时,UART硬件请求RISC-V产生一个中断,然后激活xv6的自陷handler。
自陷handler会调用\texttt{devintr}\href{https://github.com/mit-pdos/xv6-riscv/blob/risc/kernel/trap.c#L178}{(kernel/trap.c:178)},它会查看RISC-V的\texttt{scause}寄存器然后发现这个中断来自于一个外部设备。
然后它会请求一个叫PLIC[3]的硬件告诉它是哪个设备的中断\href{https://github.com/mit-pdos/xv6-riscv/blob/risc/kernel/trap.c#L187}{(kernel/trap.c:187)}。
如果是UART的话,\texttt{devintr}会调用\texttt{uartintr}。

\texttt{uartintr}\href{https://github.com/mit-pdos/xv6-riscv/blob/risc/kernel/uart.c#L176}{(kernel/uart.c:176)}从UART硬件中读取所有就绪的输入字符,然后把它们交给\texttt{consoleintr}\href{https://github.com/mit-pdos/xv6-riscv/blob/risc/kernel/console.c#L136}{(kernel/console.c:136)}处理,它并不会等待字符,因为之后的输入将会产生一个新的中断。
\texttt{consoleintr}的工作是往累积\texttt{cons.buf}中累积输入字符直到一行结束。
\texttt{consoleintr}会特殊对待退格和少数其他字符。
当一个换行符到达时,\texttt{consoleintr}会唤醒一个正在等待的\texttt{consoleread}(如果有的话)。

被唤醒后,\texttt{consoleread}会看到\texttt{cons.buf}中有一整行,然后把它拷贝到用户空间,并(通过系统调用机制)返回到用户空间。

\section{代码:控制台输出}
对连接到控制台的文件描述符的\texttt{write}系统调用最终会到达\texttt{uartputc}\href{https://github.com/mit-pdos/xv6-riscv/blob/riscv//kernel/uart.c#L87}{(kernel/uart.c:87)}。
设备驱动会维护一个输出缓冲区(\texttt{uart\_tx\_buf}),这样写入操作不需要等待UART完成发送。
事实上,\texttt{uartputc}把每个字符附加到缓冲区里,调用\texttt{uartstart}来开始设备传输(如果不是已经在传输了的话),然后返回。
\texttt{uartputc}唯一会等待的情况是缓冲区已经满了。

UART每发送完成一个字节,就会产生一个中断。
\texttt{uartintr}调用\texttt{uartstart},后者会检查设备是否已完成发送,并把缓冲区里下一个要发送的字符传给设备。
因此如果一个进程往控制台写入了很多字节,通常第一个字节会被\texttt{uartputc}调用的\texttt{uartstart}发送出去,剩下的字节将会被\texttt{uartintr}(当传输完成中断到来时会触发它)调用的\texttt{uartstart}发送出去。

需要注意的一般模式是通过缓冲和中断将设备活动与进程活动解耦。
即使没有进程在等待读取,控制台驱动也可以处理输入,之后的读取将可以看到这些输入。
类似地,进程可以无需等待设备直接发送输出。
这种解耦可以让进程和设备I/O并行执行,进而提高性能。
当设备很慢(例如UART)或者需要立即关注(例如回显输入的字符)的时候这一点尤其重要。
这个思路有时被称为\emph{I/O并发(I/O concurrency)}。

\section{驱动中的并发}
你可能已经注意到了\texttt{consoleread}和\texttt{consoleintr}中的\texttt{acquire}。
这些调用获取一个锁,使用这个锁在并发时保护控制台驱动的数据结构。
这里有三个并发风险:不同CPU上的两个进程可能会同时调用\texttt{consoleread};硬件可能在CPU已经正在执行\texttt{consoleread}时要求CPU产生一个控制台(实际是UART)中断;硬件也可能在\texttt{consoleread}正在执行时在另一个CPU上产生一个控制台中断。
这些风险可能会导致竞争或者死锁。
\autoref{ch06}会探索这些问题以及锁如何解决它们。

驱动中另一种需要关注的并发情况是一个进程可能会等待设备的输入,但输入的中断信号可能会在另一个进程(或者没有进程)正在进行时到达。
因此中断处理程序没有办法访问被中断的进程或代码。
例如,中断处理程序不能用当前进程的页表安全地调用\texttt{copyout}。
中断处理程序通常只做很少的工作(例如,只把输入数据拷贝到一个缓冲区),然后唤醒上半段代码来进行剩余的处理。

\section{时钟中断}
xv6使用时钟中断来维护它的时钟,并允许xv6在进程间切换。
\texttt{usertrap}和\texttt{kerneltrap}中的\texttt{yield}会导致进程切换。
时钟中断来自于连接到RISC-V CPU的时钟。
xv6对时钟硬件进行编程以周期性的中断每个CPU。

RISC-V要求时钟中断要在机器模式下处理,而不是管理模式。
RISC-V机器模式在没有分页的情况下运行,还有一组单独的控制寄存器,因此想在机器模式下运行普通的xv6内核代码是不切实际的。
因此,xv6对时钟中断进行了单独处理,与上面的自陷机制完全分离开来。

\texttt{start.c}中的这段代码\href{https://github.com/mit-pdos/xv6-riscv/blob/riscv//kernel/start.c#L63/start.c:63}{(kernel/start.c:63)}会在机器模式下执行,它会在\texttt{main}之前执行,负责设置接收时钟中断。
其中一部分工作是编程CLINT硬件(core-local的中断器)以让它在经过指定的延迟后生成一个中断。
另一部分是设置好一块类似于trapframe的临时存储区域,以帮助时钟中断处理程序保存寄存器和CLINT寄存器的地址。
最后,\texttt{start}把\texttt{timervec}存入\texttt{mtvec}寄存器并开启时钟中断。

用户或内核代码在执行时随时都可能出现时钟中断,内核无法在关键操作期间禁用时钟中断。
因此时钟中断处理程序必须以一种保证不会扰乱被中断的内核代码的方式完成自己的工作。
处理程序的基本策略是让RISC-V抛出一个“软件中断”然后立即返回。

RISC-V通过普通的自陷机制把软件中断传递给内核,并允许内核禁用它们。
处理时钟中断生成的软件中断的代码可以在\texttt{devintr}\href{https://github.com/mit-pdos/xv6-riscv/blob/riscv//kernel/trap.c#L205}{(kernel/trap.c:205)}中找到。

机器模式下的时钟中断处理程序是\texttt{timervec}\href{https://github.com/mit-pdos/xv6-riscv/blob/riscv//kernel/kernelvec.S#L95}{(kernel/kernelvec.S:95)}。
它在\texttt{start}准备的临时存储区域中保存几个寄存器、告诉CLINT什么时候生成下一个时钟中断、让RISC-V抛出一个软件中断、恢复寄存器、返回。
时钟中断处理程序中没有C代码。

\section{练习}
\begin{enumerate}
    \item 修改\texttt{uart.c},让它完全不使用中断。你可能还需要修改\texttt{console.c}。
    \item 为以太网卡添加一个驱动。
\end{enumerate}

